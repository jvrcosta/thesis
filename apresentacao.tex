\documentclass[11pt]{beamer}
\usetheme{Warsaw}
\usepackage[utf8]{inputenc}
\usepackage[brazil]{babel}
\usepackage[T1]{fontenc}
\usepackage{amsmath}
\usepackage{amsfonts}
\usepackage{amssymb}
\usepackage{natbib}

\theoremstyle{nonumberplain}
%\theorempostskip{10pt}
%\theoremseparator{:}
%\theoremsymbol{\ensuremath{_\square}}
%\newtheorem{proof}{Dem.}

\theoremstyle{plain}
%\theoremseparator{~}
%\theoremsymbol{}
%\newtheorem{theorem}{Teorema}


\author[João Vítor Rego Costa]{João Vítor Rego Costa \\ Orientador: Prof. Dr. Gil Riella}
\title[Menus com Contingências Imprecisas]{Uma Demonstração Alternativa para
Representação de Preferências sobre Menus com
Contingências Imprecisas}
%\setbeamercovered{transparent} 
%\setbeamertemplate{navigation symbols}{} 
%\logo{} 
%\institute{} 
%\date{} 
%\subject{} 
\begin{document}

\begin{frame}
\titlepage
\end{frame}

\begin{frame}{Axiomas da Preferência I}
Seja $B$ um conjunto finito de alternativas e $\Delta(B)$ o conjunto das medidas de probabilidade sobre $B$. $\mathbb{X}$ é a coleção de subconjuntos fechados de $\Delta(B)$, os menus, e $\succsim$ denotará a preferência sobre $\mathbb{X}$.

\vspace{12pt}

\begin{description}
\item [\textit{Order}] $\succsim$ é completa e transitiva
\item [\textit{Continuity}] $\forall x\in \mathbb{X}$, $\{y\in \mathbb{X}:y\succsim x\}$ e $\{y\in \mathbb{X}:x\succsim y\}$ são fechados

\item [\textit{Monotonicity}] Para quaisquer $x$, $x'$ $\in\mathbb{X}$ com $x\begin{small}\supseteq\end{small} x'$, temos $x\succsim x'$

\item [\textit{Indifference to Randomization (IR)}] $x\sim co(x)$

\item [\textit{Nondegeneracy}] Existem menus $x$,$x'$ $\in \mathbb{X}$ tais que $x\succ x'$

\item [\textit{Preference Convexity}] $x\succsim x' \Rightarrow \lambda x +(1-\lambda)x'\succsim x'$.
\end{description}
\end{frame}

\begin{frame}{Axiomas da Preferência II}
Para qualquer estado, $\{b_*\}$ e $\Delta(B)$ proporcionam o menor e maior \emph{payoffs}, respectivamente.
 
\vspace{10pt}

Seja, então, o menu certo $x_p:=p\Delta(B)+(1-p)\{b_*\}$

\vspace{10pt}

\begin{description}
\item [\textit{Certainty Independence}] Para $\lambda\in (0,1)$ e $x_p$, temos $$x\succsim x' \Leftrightarrow \lambda x +(1-\lambda)x_p\succsim \lambda x' + (1-\lambda)x_p$$
\end{description}
\pause
\begin{description}
\item[\textit{Finiteness}] Para todo $x$, existe um menu finito $x^f\subseteq x$ tal que, para todo $\lambda\in (0,1]$ e qualquer menu $x'$, $\lambda x +(1-\lambda)x' \sim \lambda x^f +(1-\lambda)x'$.
\item[\textit{Worst}] Para a pior alternativa $b_*$, ~temos $\lambda\left(x\cup \{b_*\}\right)+(1-\lambda)y\sim \lambda x + (1-\lambda)y$ para quaisquer menus $x,y\in \mathbb{X}$ e $\lambda\in (0,1)$.
\end{description}

\end{frame}

\begin{frame}{Resultado principal}
\begin{block}{Teorema 1}
A preferência $\succsim$ sobre o espaço de menus $\mathbb{X}$ satisfaz os axiomas mencionados acima se, e somente se, existe um conjunto finito de utilidades $N\subseteq \{u \in \mathbb{R}^B_+:u(b_*)=0\text { e }\max_{B}u(b)=1\}$ e um conjunto fechado e convexo $\Pi$ de medidas de probabilidade sobre $N$ tais que 
\[x \succsim y \;\Leftrightarrow\; \min_{\pi\in\Pi}\sum_{u\in N} \pi(u)\max_{\beta\in x}\mathbb{E}_\beta(u)\geq \min_{\pi\in\Pi}\sum_{u\in N} \pi(u)\max_{\beta\in y}\mathbb{E}_\beta(u)\]
para quaisquer $x,y\in\mathbb{X}$ e $\mathbb{E}_\beta(u)$ a utilidade esperada vNM
da loteria $\beta$. 
\end{block} 
\end{frame}

\begin{frame}{Duas abordagens}
Nosso trabalho propõe uma demonstração alternativa àquela encontrada em \cite{Epstein2007}.

\vspace{10pt}

\begin{description}
\item[EMS] Como a preferência satisfaz \emph{Certainty Independence}, (i) começa-se obtendo uma representação para os menus certos para, em seguida, (ii) estendê-la a todo o espaço de menus.

\pause
\vspace{10pt}

\item[Alternativa] (i) Definimos $\succsim^*$ como o maior subconjunto de $\succsim$ que satisfaz independência; (ii) apesar de $\succsim^*$ ser incompleta, podemos... 
\end{description}
\end{frame}

\begin{frame}{Representação para preferências incompletas}
... utilizar o resultado abaixo; 
\begin{block}{Teorema 2 (\cite{Kochov2007})}
Uma preordem $\succcurlyeq\subseteq\mathbb{X}\times\mathbb{X}$ satisfaz \emph{Continuity}, \emph{Nondegeneracy}, \emph{Independence} e \emph{Monotonicity} se, e somente se, existe um conjunto $S$, uma função utilidade dependente de estado $U:\Delta(B)\times S\rightarrow R$ e um conjunto fechado e convexo $\mathcal{M}$ de medidas de probabilidade sobre $S$ tais que
\begin{enumerate}[(i)]
\item $x\succcurlyeq y$ se, e somente se, $$ \int_{S} \max_{\beta\in x}U(\beta,s)d\mu \geq \int_{S} \max_{\beta\in y}U(\beta,s)d\mu\quad \forall\mu\in\mathcal{M};$$
\item cada $U(\cdot,s)$ é uma função utilidade esperada, i.e. $$U(\beta,s)=\sum_{b\in B} \beta(b)U(b,s).$$
\end{enumerate}  
\end{block}
\end{frame}

\begin{frame}{Representação para preferências incompletas}
(iii) adaptamos o Teorema 2 ao caso de $\succsim^*$,  provando a finitude do espaço de estados e normalizando as utilidades;

\begin{block}{Lema 1}
Existe um conjunto finito de funções $N\subseteq\{u \in \mathbb{R}^B_+:u(b_*)=0\text { e }\max_{B}u(b)=1\}$ e um conjunto fechado e convexo $\Pi$ de medidas de probabilidade sobre $N$ tais que, para todo $x,y\in\mathbb{X}$:
$$x\succsim^* y \Leftrightarrow \sum_{u\in N} \pi(u)\max_{\beta\in x}\mathbb{E}_\beta(u) \geq \sum_{u\in N} \pi(u)\max_{\beta\in y}\mathbb{E}_\beta(u)\quad \forall\pi\in\Pi$$
\end{block}

\end{frame}

\begin{frame}{\emph{Negative Certainty Independence (NCI)}}
(iv) demonstramos que a relação original satisfaz NCI. Por fim, basta obter o formato min-max do Teorema 1.
\begin{block}{Lema 2}
A relação $\succsim$ satisfaz \emph{Negative Certainty Independence (NCI)}, i.e. se $x\succsim x_p$, então $\lambda x +(1-\lambda)y\succsim \lambda x_p +(1-\lambda)y$ para todo $\lambda\in(0,1)$ e $y\in \mathbb{X}$.
\end{block}

\end{frame}

\begin{frame}{Aplicações}
Propomos duas aplicações da técnica que utilizamos para obter resultados do mundo de \emph{atos} adaptados para o caso de menus:
\begin{description}
\item[Preferências Variacionais (MMR)]
Adaptaríamos a axiomatização Bewley-variacional de \cite{Faro2015} para o caso de menus e, em seguida, aplicaríamos a demonstração de \cite{Brotherhood2014} para obter a representação variacional de \cite{Maccheroni2006}.

\item[Racionalidade Objetiva e Subjetiva (GMMS)]
Adicionando dois axiomas, \emph{Consistency} e \emph{Default to Certainty}, às primitivas do modelo de \cite{Gilboa2010}, podemos obter uma versão do resultado principal dos autores para o caso de menus.
\end{description}

\end{frame}

\begin{frame}{Aplicações II}
Um terceira aplicação possível se refere à prova de que os agentes são \emph{bayesian updaters} de suas crenças quando recebem um sinal a respeito do espaço de estados da natureza (\cite{Riella2013a}).\pause
\begin{block}{Definição: \emph{Positive Additive Expected Utility (PAEU)}}
Dizemos que uma relação $\succcurlyeq\subseteq\mathbb{X}\times\mathbb{X}$ é PAEU se existe um conjunto $N\subseteq \{u \in \mathbb{R}^B_+:u(b_*)=0\text { e }\max_{B}u(b)=1\}$ e um conjunto fechado e convexo $\Pi$ de medidas de probabilidade sobre $N$ tais que:
\begin{enumerate}[1.]
\item $x \succcurlyeq y \;\Leftrightarrow\;$ $$\min_{\pi\in\Pi}\sum_{u\in N} \pi(u)\max_{\beta\in x}\mathbb{E}_\beta(u)\geq \min_{\pi\in\Pi}\sum_{u\in N} \pi(u)\max_{\beta\in y}\mathbb{E}_\beta(u)$$
\item $\bigcup_{\pi\in\Pi}supp(\pi)=N$ e, para $u$ e $u'$ distintas, podemos afirmar que não são uma transformação positiva afim uma da outra.
\end{enumerate}
\end{block} 
\end{frame}

\begin{frame}{Aplicações II}
Agora, considere duas preferências PAEU, $\succcurlyeq$ e $\succcurlyeq^*$. Como fizemos anteriormente, identifique o maior subconjunto dessas relações que satisfazem independência, $\succcurlyeq_r$ e $\succcurlyeq^*_r$, que são incompletas. Impondo o axioma de \emph{Flexibility Consistency} de \cite{Moura2013}...
\begin{block}{\emph{Flexibility Consistency}}
Para quaisquer menus $x,y\in \mathbb{X}$, $x\succcurlyeq^*_r y$ e não $x\succcurlyeq_r y$ ou $y\succcurlyeq^*_r x$ e não $y\succcurlyeq_r x$ implicam que existe um menu $z$ tal que $x\cup y\cup z \sim^*_r x\cup z$, mas $x\cup y\cup z \succ_r x\cup z$.
\end{block} 
\end{frame}

\begin{frame}
...obtemos que as seguintes afirmações são equivalentes
\begin{enumerate}[1.]
\item Sejam $N$ e $N^*$ os espaços de estados subjetivos de $\succcurlyeq_r$ e $\succcurlyeq^*_r$, respectivamente. Para quaisquer menus $x$ e $y$ com 
$$\max_{\beta\in x}\mathbb{E}_\beta(u) = \max_{\beta\in y}\mathbb{E}_\beta(u)\;\forall u\in N\setminus N^*,$$
$x\succcurlyeq_r y \Leftrightarrow y\succcurlyeq^*_r x$.
\item Para toda representação $(N,\Pi)$ de $\succcurlyeq_r$, existe $M\subseteq N$ tal que $(M,\Pi_M)$ representa $\succcurlyeq_r^*$, onde $\Pi_M$ é o conjunto de \emph{priors} $\pi\in \Pi$ com $\pi(M)>0$ atualizadas pela regra de Bayes.
\end{enumerate} 
\end{frame}

\begin{frame}
\scriptsize
\bibliography{library}
\bibliographystyle{chicago}
\end{frame}
\end{document}
