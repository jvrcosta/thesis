\documentclass[11pt]{beamer}
\usetheme{Warsaw}
\usepackage[utf8]{inputenc}
\usepackage[brazil]{babel}
\usepackage[T1]{fontenc}
\usepackage{amsmath}
\usepackage{amsfonts}
\usepackage{amssymb}
\author[João Vítor Rego Costa]{João Vítor Rego Costa \\ Orientador: Prof. Dr. Gil Riella}
\title[Menus com Contingências Imprecisas]{Uma Demonstração Alternativa para
Representação de Preferências sobre Menus com
Contingências Imprecisas}
%\setbeamercovered{transparent} 
%\setbeamertemplate{navigation symbols}{} 
%\logo{} 
%\institute{} 
%\date{} 
%\subject{} 
\begin{document}

\begin{frame}
\titlepage
\end{frame}

%\begin{frame}
%\tableofcontents
%\end{frame}

\begin{frame}{Motivação}

\end{frame}

\begin{frame}{Axiomas da Preferência I}
Seja $B$ um conjunto finito de alternativas e $\Delta(B)$ o conjunto das medidas de probabilidade sobre $B$. $\mathbb{X}$ é a coleção de subconjuntos fechados de $\Delta(B)$, os menus, e $\succsim$ denotará a preferência sobre $\mathbb{X}$.

\vspace{12pt}

\begin{description}
\item [\textit{Order}] $\succsim$ é completa e transitiva
\item [\textit{Continuity}] $\forall x\in \mathbb{X}$, $\{y\in \mathbb{X}:y\succsim x\}$ e $\{y\in \mathbb{X}:x\succsim y\}$ são fechados

\item [\textit{Monotonicity}] Para quaisquer $x$, $x'$ $\in\mathbb{X}$ com $x\begin{small}\supseteq\end{small} x'$, temos $x\succsim x'$

\item [\textit{Indifference to Randomization (IR)}] $x\sim co(x)$

\item [\textit{Nondegeneracy}] Existem menus $x$,$x'$ $\in \mathbb{X}$ tais que $x\succ x'$

\item [\textit{Preference Convexity}] $x\succsim x' \Rightarrow \lambda x +(1-\lambda)x'\succsim x'$.
\end{description}
\end{frame}

\begin{frame}{Axiomas da Preferência II}
Para qualquer estado, $\{b_*\}$ e $\Delta(B)$ proporcionam o menor e maior \emph{payoffs}, respectivamente.
 
\vspace{10pt}

Seja, então, o menu certo $x_p:=p\Delta(B)+(1-p)\{b_*\}$

\vspace{10pt}

\begin{description}
\item [\textit{Certainty Independence}] Para $\lambda\in (0,1)$ e $x_p$, temos $$x\succsim x' \Leftrightarrow \lambda x +(1-\lambda)x_p\succsim \lambda x' + (1-\lambda)x_p$$
\end{description}
\pause
\begin{description}
\item[\textit{Finiteness}] Para todo $x$, existe um menu finito $x^f\subseteq x$ tal que, para todo $\lambda\in (0,1]$ e qualquer menu $x'$, $\lambda x +(1-\lambda)x' \sim \lambda x^f +(1-\lambda)x'$.
\item[\textit{Worst}] Para a pior alternativa $b_*$, ~temos $\lambda\left(x\cup \{b_*\}\right)+(1-\lambda)y\sim \lambda x + (1-\lambda)y$ para quaisquer menus $x,y\in \mathbb{X}$ e $\lambda\in (0,1)$.
\end{description}

\end{frame}

\begin{frame}{Resultado pricipal}
.
\end{frame}

\begin{frame}{Dois \emph{approaches}}
.
\end{frame}

\begin{frame}{Representação para preferências incompletas}
.
\end{frame}

\begin{frame}{\emph{Negative Certainty Independence (NCI)}}
.
\end{frame}

\begin{frame}{Aplicações}
.
\end{frame}

\begin{frame}{Aplicações II}
.
\end{frame}
\end{document}